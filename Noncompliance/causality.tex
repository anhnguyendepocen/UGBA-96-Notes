\documentclass{article}
\usepackage[utf8]{inputenc}
\usepackage{amsmath}
\title{Noncompliance}
\author{Alan Liang}
\date{May 2018}

\begin{document}

\maketitle

\section{Motivation and Goal}

Sometimes, we do not always have full control over who receives the treatment: it is often unethical and/or infeasible to mandate or exclude participants from receiving the treatment.
For example, when conducting an experiment on the effects of charter schools on standardized test scores, it'd be infeasible to prevent or force certain participants to being enrolled in a charter school. 
\\
Furthermore, if we were to compare the naturally forming treatment and control groups, then the samples would not be randomly chosen, and hence would not reflect the true treatment effect. 
For example, those who are determined to go to charter school even if they were assigned otherwise are perhaps more likely to place a larger emphasis on education and hence attain higher test scores.
\\
\\
When conducting an experiment, noncompliance is a threat to the experiment's validity.
Noncompliance is the phenomena in which participants assigned to their assigned treatment/control groups may not comply to stay within their assignment, instead opting for the opposite. 

\section{Terminology and Background} 


\end{document}
