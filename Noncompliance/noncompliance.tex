\documentclass{article}
\usepackage[utf8]{inputenc}
\usepackage{amsmath}
\addtolength{\oddsidemargin}{-0.625in}
\addtolength{\evensidemargin}{-0.625in}
\addtolength{\textwidth}{1.5in}
\addtolength{\topmargin}{-1in}
\addtolength{\textheight}{1.5in}
\title{Noncompliance}
\author{Alan Liang}
\date{May 2018}

\begin{document}

\maketitle

\section{Motivation and Goal}

Sometimes, we do not always have full control over who receives the treatment: it is often unethical and/or infeasible to mandate or exclude participants from receiving the treatment.
For example, when conducting an experiment on the effects of charter schools on standardized test scores, it'd be infeasible to prevent or force certain participants to being enrolled in a charter school. 
\\
Furthermore, if we were to compare the naturally forming treatment and control groups, then the samples would not be randomly chosen, and hence would not reflect the true treatment effect. 
For example, those who are determined to go to charter school even if they were assigned otherwise are perhaps more likely to place a larger emphasis on education and hence attain higher test scores.
\\
\\
Noncompliance is the phenomena in which participants assigned to their assigned treatment/control groups may not comply to stay within their assignment, instead opting for the opposite. 
When conducting an experiment, noncompliance is a threat to the experiment's validity: it dilutes hte relationship between the treatment assignment and the treatment received.

\section{Terminology} 
\subsection{Notation and Experiment Setup}
Noncompliance occurs when we cannot mandate or exclude participants from gaining the treatment.
This generally occurs through experiments in which individuals are selected to be part of the treatment, but are not forced to join.
In addition, it also does not necessarily mean that individuals that are not selected cannot gain the treatment. 
Examples of these experiments include winning a lottery for attending charter schools or free healthcare, being selected to certain store promotions, and more. 
It is important to keep in mind that in the examples above, attending charter schools or having free healthcare is the \textit{treatment variable}, which is what we are interested in determining the causal effect of on the outcome variable. 
\\
Our variables are denoted as follows: 
\begin{itemize}
    \item \textbf{Outcome}: a random variable denoted by $Y_i$ for a particular individual $i$.
    \item \textbf{Treatment}: a random variable denoted by $T_i$ for a particular individual $i$.
\end{itemize}
In addition, the \textbf{instrument variable} $Z_i$ is the random variable which defines whether the individual is selected to be part of the treatment group (e.g. winning the lottery).


\subsection{Types of participants}
There are 4 types of participants in an experiment: they are separated by what they would do if they received the instrument variable or not (whether they won the lottery):
\begin{center}
    \begin{tabular}{c|ccc}
        &&Lottery Losers&($Z_i = 0$) \\ 
        \hline
         & &No Treatment($T_i=0$)&Treatment($T_i=1$) \\
         Lottery Winners &No Treatment ($T_i=0$)&Never Takers&Defiers  \\
         ($Z_i = 1$) &Treatment ($T_i=1$)&Compliers&Always Takers \\
    \end{tabular}
\end{center}
A \textbf{never-taker} is a participant who will never take the treatment variable.
For example, a never-taker in the charter school experiment is a participant who will never enroll in charter school, regardless if they won the lottery or not.
\\
A \textbf{complier} is a participant who will take the treatment variable if and only if he/she received the instrument variable. 
These are the participants who enrolled in charter school if they won the lottery, and won't if they didn't win.
\\
An \textbf{always-taker} is a participant who will always take the treatment variable.
They will enroll in charter no matter of outcome in winning the lottery.
\\
A \textbf{defier} is a participant who will go against the instrument variable. 
They will only enroll in charter schools if they don't win the lottery, and won't enroll in charter schools if they wins the lottery.
Normally, defiers are assumed to not exist, according to the monotonocity assumption (see below). 
\\\\
Notably, because we can only observe the partial outcomes, we do not know what type each participant is! 
For example, t group that receives the instrument will have all 4 types of participants in it; 2 of the groups–the always takers and compliers–will take the treatment, and the other 2–the never takers and defiers–will not.

\section{Local Average Treatment Effect}
\subsection{Relationship of the Instrument, Treatment, and Outcome Variables}
All of our calculations are based on the following assumption: 
\begin{gather*}
\textrm{Observed effect of instrument on outcome} = \\\textrm{Observed effect of instrument on treatment} \times \textrm{Effect of treatment on outcome}
\end{gather*}
We often call the effect of instrument on treatment the first stage, and the effect of treatment on outcome the second stage. Intuitively, if the first stage is near 0, then the instrument variable will not affect the treatment and hence should not affect the outcome.
On the other hand, if the second stage is near 0, then the treatment will not affect the outcome, and hence the instrument should not affect the outcome either. 
Together, these 2 intuitions form the basis of the exclusion restriction assumption (see below).  


\subsection{Calculating the Local Average Treatment Effect}
The effect of the treatment on the outcome is what we are ultimately interested in to determine causality.
Hence, from the equation above, we can solve for the effect of treatment on outcome:
\begin{gather*}
\textrm{Effect of treatment on outcome} = \frac{\textrm{Observed effect of instrument on outcome}}{\textrm{Observed effect of instrument on treatment}}
\end{gather*}
The consequent effect of treatment on outcome, known as the \textbf{local average treatment effect (LATE)}, measures the treatment effect for compliers only. 
With noncompliance, we are not able to directly identify the average treatment effect, and will have to make do with the LATE.
\\
Another way to build intuition is by thinking that we essentially utilize the instrument variable to determine its effects on both the treatment variable and the outcome variable to determine the effect of treatment on the outcome:
\begin{gather*}
\frac{\text{Change in outcome due to instrument}}{\text{Change in treatment due to instrument}} = \frac{\text{change in outcome} / \text{change in instrument}}{\text{change in treatment} / \text{change in instrument}} \\= \frac{\text{Change in outcome}}{\text{Change in treatment}} = \text{Effect of treatment on outcome} \\
\frac{\Delta Y_i/\Delta Z_i}{\Delta T_i/ \Delta Z_i} = \frac{\Delta Y_i}{\Delta T_i}
\end{gather*}
Note that the derivation and definitions above are not particularly rigorous, but I hope it builds enough intuition as to why this calculation works.

\subsection{Calculating the Two Stages}
\textbf{1st Stage}: The observed effect of the instrument variable on the treatment variable is the difference between the proportion that took the treatment in the group that received the instrument variable and the proportion that took the treatment in the group that did not:
\\
\centerline{$E[Y_i|Z_i=1]-E[Y_i|Z_i=0]$}
\textbf{2nd Stage}: The observed effect of the instrument variable on the outcome is also known as the \textbf{intent-to-treat effect (ITT)}. 
It describes the difference between the average outcome of the group that received the instrument variable and the average outcome of the group that did not:
\\
\centerline{$E[T_i|Z_i=1]-E[T_i|Z_i=0]$}
\\
We can then combine these 2 results by dividing first by the second to get the LATE:
\begin{gather*}
\textrm{LATE} = \frac{\textrm{Observed effect of instrument on outcome}}{\textrm{Observed effect of instrument on treatment}} = \frac{E[Y_i|Z_i=1]-E[Y_i|Z_i=0]}{E[T_i|Z_i=1]-E[T_i|Z_i=0]}
\end{gather*} 


\subsection{Assumptions}
The above equations indicate a few assumptions:
\begin{itemize}
    \item \textit{Relevance}: the instrument variable must have a causal effect on the treatment. For example, winning the lottery should have a causal effect on attending charter schools.
    \item \textit{Independence}: the instrument variable should be randomly assigned.
    \item \textit{Exclusion Restriction}: the instrument variable does not directly affect the outcome. This relates to the links above: if the second link is near 0, then the observed instrument variable's effect on the outcome should be near 0 also.
    \item \textit{Monotonicity}: there are no defiers in the participants. 
\end{itemize}

\subsection{A Special Case: LATE = TOT}
In some experiments, participants cannot receive the treatment unless they receive the instrument variable, so that there are no always-takers. 
An example of this would be receiving a certain promotion for a sale, in which participants who did not receive the promotion cannot participate in the sale. 
In this case, since there are no always-takers, the LATE is the same as the treatment on the treated (TOT), as only the compliers–those who were randomly assigned to receive the treatment–received the treatment. 
\end{document}
