\documentclass{article}
\usepackage[utf8]{inputenc}
\usepackage{amsmath}
\title{Noncompliance}
\author{Alan Liang}
\date{May 2018}

\begin{document}

\maketitle

\section{Motivation and Goal}

Sometimes, we do not always have full control over who receives the treatment: it is often unethical and/or infeasible to mandate or exclude participants from receiving the treatment.
For example, when conducting an experiment on the effects of charter schools on standardized test scores, it'd be infeasible to prevent or force certain participants to being enrolled in a charter school. 
\\
Furthermore, if we were to compare the naturally forming treatment and control groups, then the samples would not be randomly chosen, and hence would not reflect the true treatment effect. 
For example, those who are determined to go to charter school even if they were assigned otherwise are perhaps more likely to place a larger emphasis on education and hence attain higher test scores.
\\
\\
When conducting an experiment, noncompliance is a threat to the experiment's validity.
Noncompliance is the phenomena in which participants assigned to their assigned treatment/control groups may not comply to stay within their assignment, instead opting for the opposite. 

\section{Terminology} 
\subsection{Notation and Experiment Setup}
Noncompliance occurs when we cannot mandate or exclude participants from gaining the treatment.
This generally occurs through experiments in which individuals are selected to be part of the treatment, but are not forced to join.
In addition, it also does not necessarily mean that individuals that are not selected cannot gain the treatment. 
Examples of these experiments include winning a lottery for attending charter schools or free healthcare, being selected to certain store promotions, and more. 
It is important to keep in mind that in the examples above, attending charter schools or having free healthcare is the \textit{treatment variable}, which is what we are interested in determining the causal effect of on the outcome variable. 
\\
Our variables are denoted as follows: 
\begin{itemize}
    \item \textbf{Outcome}: a random variable denoted by $Y_i$ for a particular individual $i$.
    \item \textbf{Treatment}: a random variable denoted by $T_i$ for a particular individual $i$.
\end{itemize}
In addition, the \textbf{instrument variable} $Z_i$ is the random variable which defines whether the individual is selected to be part of the treatment group (e.g. winning the lottery).

\subsection{Types of participants}
There are 4 types of participants in an experiment: they are separated by what they would do if they received the instrument variable or not (whether they won the lottery):
\begin{center}
    \begin{tabular}{c|ccc}
        &&Lottery&Losers \\ 
        \hline
         & &No Treatment($T_i=0$)&Treatment($T_i=1$) \\
         Lottery&No Treatment ($T_i=0$)&Never Takers&Defiers  \\
         Winners &Treatment ($T_i=1$)&Compliers&Always Takers \\
    \end{tabular}
\end{center}
Notably, because we can only observe the partial outcomes, we do not what type each participant is!



\section{Local Average Treatment Effect}
\subsection{Calculations}

\subsection{Assumptions}
There are 4 main assumptions for 
\end{document}
